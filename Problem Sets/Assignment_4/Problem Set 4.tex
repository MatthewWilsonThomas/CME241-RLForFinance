\documentclass[11pt]{article}

% Load package
\usepackage{../lesson}
\setcounter{MaxMatrixCols}{20}

% Set title and course name
\settitle{Problem Set 4}
\setsubtitle{Utility Functions and Risk Aversion}
\setcourse{CME241 - Reinforcement Learning for Finance}
\setdate{Due: 01.02.2023}

\begin{document}

% Create title and add proper header for first page
\maketitle
\thispagestyle{first}

\section{Approximate Policy Iteration}


%----------------------------------------------
\newpage
\section{Utility Functions}
\paragraph*{Expected Utility}
\begin{align}
    \E{U(x)} &= \E{x - \frac{\alpha}{2}x^2} \\
    &= \E{x} - \frac{\alpha}{2}\E{x^2} \\
    &= \mu - \frac{\alpha}{2}(\sigma^2+\mu^2)
\end{align}
\paragraph*{Certainty Equivalent}
\begin{align}
    U(x_{CE}) &= \E{U(x)}\\
    x - \frac{\alpha}{2} &= \mu - \frac{\alpha}{2}(\sigma^2 + \mu^2) \\
    \text{Solving using quadratic formula gives:}\quad &\\
    x &= \frac{1\pm\sqrt{\alpha^2(\mu^2 + \sigma^2)-2\alpha\mu + 1}}{\alpha}
\end{align}

\paragraph*{Absolute Risk Premium}
The ARP is defined as:
\begin{align}
    \pi_A &= \E{x} - x_{CE} \\
    &= \mu - \frac{1\pm\sqrt{\alpha^2(\mu^2 + \sigma^2)-2\alpha\mu + 1}}{\alpha}
\end{align}

\paragraph*{Optimal investment rate}
To optimize the utility given the risk aversion we can find the distribution of wealth at the end of the year (given the investment choice):
\begin{align}
    W_1 &\sim \mathcal{N}(z\mu + (1-z)r, z^2\sigma^2) \\
    \E{U(W_1)} &= z\mu + (1-z)r - \frac{\alpha}{2}(z^2\sigma^2 + (z\mu + (1-z)r)^2)
\end{align}
We can optimize this through:
\begin{align}
    \frac{\partial^2}{\partial f^2}\E{U(W_1)} &= \mu - r - \alpha(z\sigma^2 + z\mu^2 - r^2 + + zr^2 + \mu r - 2z\mu r)  = 0\\
    z &= \frac{(\mu - r)(\frac{1}{\alpha} - r)}{\sigma^2 + \mu^2 + r^2 - 2\mu r}
\end{align}


%----------------------------------------------
\newpage
\section{Kelly Criterion}
Starting with a bet of $fW_0$ dollars you have the following behaviour:
\begin{align}
    W_1 &= (1 - f)W_0 + \begin{cases}
        fW_0(1 + \alpha);\quad p \\
        fW_0(1 - \beta);\quad 1-p
    \end{cases} \\
    &= W_0 + \begin{cases}
        fW_0\alpha; \quad p \\
        -fW_0\beta; \quad 1-p
    \end{cases} \\
    &= W_0 \cdot \begin{cases}
        (1 + f\alpha);\quad p\\
        (1 - f\beta); \quad 1-p
    \end{cases}
\end{align}.

Then we know that the two outcomes for log wealth can be written as:
\begin{align}
    \log W_0 &= \log W_0 + \begin{cases}
        (1 + f\alpha);\quad p\\
        (1 - f\beta); \quad 1-p
    \end{cases}\\
    \text{Giving: }\quad &\\
    \E{\log W_1} &= \log W_0 + p\log (1 + f\alpha) + (1-p)\log(1 - f\beta)
\end{align}
Taking the derivative w.r.t $f$ results in the following:
\begin{align}
    \frac{\partial}{\partial f}\E{\log W_1} &= \frac{p\alpha}{1 + f\alpha} - \frac{(1-p)\beta}{1-f\beta} \\
    \text{Giving: }\quad&\\
    f^* &= \frac{p\alpha - (1-p)\beta}{\alpha\beta} \\
    &= \frac{p}{\beta} - \frac{1-p}{\alpha}
\end{align}

This is know to be the Kelly Criterion formula as required. 

We can also see that this is the maxima by evaluating the second derivative at $f^*$:
\begin{align}
    \frac{\partial^2}{\partial f^2}\E{\log W_1} &= -\left[\frac{p\alpha^2}{(1+f\alpha)^2} + \frac{(1-p)\beta^2}{(1-p\beta)^2}\right]
\end{align}
Which is always negative, meaning we have a maxima here.

This clearly makes intuitive sense based on the dependencies on $\alpha, \beta$ and $p$. I am convinced.

\end{document}