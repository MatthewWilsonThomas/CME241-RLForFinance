\documentclass[11pt]{article}

% Load package
\usepackage{../lesson}
\usepackage{hyperref}
\setcounter{MaxMatrixCols}{20}

% Set title and course name
\settitle{Problem Set 5}
\setsubtitle{Optimal Execution}
\setcourse{CME241 - Reinforcement Learning for Finance}
\setdate{Due: 13.02.2023}

\begin{document}

% Create title and add proper header for first page
\maketitle
\thispagestyle{first}

\section*{Problem 1}

We have to minimize 
\begin{align}
g(S)&=p \cdot g_1(S)+h \cdot g_2(S)\\
&= p\int_S^{\infty}(x-S) \cdot f(x) \cdot d x + h\int_{-\infty}^S(S-x) \cdot f(x) \cdot d x
\end{align}
To minimize, we take the derivative with respect to $S$ and use the Leibnitz rule to differentiate a definite integral.
\begin{align}
\frac{dg(S)}{dS}&=p \cdot g_1'(S)+h \cdot g_2'(S)\\
&= p\int_S^{\infty}- f(x) \cdot d x + h\int_{-\infty}^Sf(x) \cdot d x\\
&= -p(1-F(S)) + h(F(S))\\
&= -p + (h+p)F(S)
\end{align}
Setting the derivative to zero, we get 
$$
S^* = F^{-1}(\frac{p}{h+p})
$$
where $F$ is the CDF of the demand function.
\newline

In terms of options, we are minimizing the expected future payoff from options underlying a stock whose price is given by the distribution $f$ and the strike price of the options are $S$. $g_1(S)$ represents a call option and $g_2(S)$ represents a put. Since we are minimizing the expected future payoff (not discounted), our portfolio can be seen to be comprised of short poisitions in $p$ calls and $h$ puts.


\section*{Problem 3}
Solution at \href{https://github.com/MatthewWilsonThomas/CME241-RLForFinance/blob/master/Problem%20Sets/Assignment_5/Q3-OrderBookDynamics.py}{Code}

\section*{Problem 4}
\begin{align}
    V_T(P_T, X_T, N_T) &= \max_{N_{t; t=T}}\{ Q_t\cdot N_t \} = R_TP_T(1-\beta R_T - \theta X_T) \\
    \text{Solving for $N^*_T$ gives:} \\
    N*_T &= R_T \\
    \text{Extending to $T-1$:} \\
    V_{T-1}(P_{T-1}, X_{T-1}, N_{T-1}) &= \max_{N_{t; t=T-1}} \{ Q_t\cdot N_t + V_{t+1}\}\\
    &= \max_{N_{t; t=T-1}}\{N_tP_t(1-\beta N_t - \theta X_t) + \E{P_TR_T(1-\beta R_T - \theta X_T)} \} \\
    R_T &= R_{T-1} - N_{T-1}; P_T = P_{T-1}e^{Z_t}; X_T = \rho X_{T-1} + \eta_{T-1} \\
    &= \max_{N_{t; t=T-1}}\{N_tP_t(1-\beta N_t - \theta X_t) \\
    &\quad+ \E{P_{T-1}e^{Z_{T-1}}(R_{T-1} - N_{T-1})(1-\beta (R_{T-1} - N_{T-1}) - \theta X_{T-1} + \eta_{T-1})} \} \\
    &= \max_{N_{t; t=T-1}}\{N_tP_t(1-\beta N_t - \theta X_t) \\
    &\quad+ q P_{T-1} (R_{T-1} - N_{T-1})(1-\beta (R_{T-1} - N_{T-1}) - \theta X_{T-1})\} \\
    q& = e^{\mu_Z + \frac{\sigma^2_Z}{2}} \\
    \text{Solving for $N^*_{T-1}$ gives:} \\
    N^*_{T-1} &= \delta_x X_{T-1} + \delta \\
    \delta_x &= \frac{\theta(\rho q - 1)}{2\beta(1 + q)} \\
    \delta &= \frac{\theta(1 - q)}{2\beta(1 + q)} \\
\end{align}
\begin{align}
    \text{This gives the Value Function as:} \\
    V_{T-1}(P_{T-1}, X_{T-1}, N_{T-1}) &= qP_{T-1}[a_1 + b_1X_{T-1} + c_1X_{T-1}^2 + d_1 R_{T-1}X_{T-1} + e_1R_{T-1} + f_1 R^2_{T-1}] \\
    a_1 &= \delta - \beta \delta^2 - q\delta - q\beta \delta^2\\
    b_1 &= \delta_x - 2\beta \delta \delta_x - \theta\delta - 2q\beta \delta\delta_x  - q\delta_x + q\theta\delta\\
    c_1 &= -\beta\delta_x^2\\
    d_1 &= q\beta\delta_x - q\theta + q\beta\delta_x\\
    e_1 &= q + 2q\beta\delta\\
    f_1 &= -q\beta\\
\end{align}
This recursion can be extended backwards to completely characterize the value function at any time step.

Implementation at \href{https://github.com/MatthewWilsonThomas/CME241-RLForFinance/blob/master/Problem%20Sets/Assignment_5/Q4-LPT.py}{code}

\end{document}
